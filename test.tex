\documentclass{ctexart}
\usepackage[utf8]{inputenc}
\usepackage[numbers,sort&compress]{natbib}
\bibliographystyle{bib/gbt7714-numerical_njust}


\usepackage{tcolorbox}
\newcommand{\latexcmd}[1]{%
    \tcbox[%  ← 这里应该是方括号 [
        on line,
        boxsep=0pt,
        left=3pt,
        right=3pt,
        top=1.5pt,
        bottom=1pt,
        colback=gray!8,
        colframe=gray!25,
        arc=4pt,
        boxrule=0.8pt,
        fontupper=\ttfamily\normalsize,
        before upper={\color{black!80}}
    ]{\texttt{\textbackslash#1}}%
}
\usepackage{xeCJK}
\usepackage{hyperref}

% 设置支持繁简体的字体
\RequirePackage{caption}

\captionsetup{
  margin=10.5pt,  
  font=small,
  labelfont=bf,
  labelsep=none,
  belowskip=10pt,   
  aboveskip=10pt 
}


\title{LaTeX 测试文档}
\author{Your Name}
\date{\today}

\begin{document}

\maketitle

\section{简介}
这是一个简单的 LaTeX \cite{zhang2020deep}测试文档。

测试不同的引用格式:
\begin{itemize}
    \item 数字引用:\cite{zhang2020deep}
    \item 括号引用:\citep{zhang2020deep}
    \item 文本内引用:\citet{zhang2020deep}
    \item 多个引用:\cite{zhang2020deep,knuth1984tex}
\end{itemize}

\noindent 说明:
\begin{itemize}
    \item \texttt{\textbackslash cite} - 数字引用,如 [1]
    \item \texttt{\textbackslash citep} - 括号引用,如 (张三 等, 2020)
    \item \texttt{\textbackslash citet} - 文本内引用,如 张三 等 (2020)
\end{itemize}

\begin{figure}[h]
    \centering
    \fbox{\rule{4cm}{3cm}}  % 带边框的方块
    \caption{测试标题}
    \label{fig:test}
\end{figure}

\subsection{数学公式测试}
行内公式:$E = mc^2$

独立公式:
\[
\int_{-\infty}^{\infty} e^{-x^2} dx = \sqrt{\pi}
\]

\subsection{列表测试}
\begin{itemize}
    \item 项目一
    \item 项目二
    \item 项目三
\end{itemize}

\section{测试繁简体显示}

正常简体中文:这是一个测试。

包含繁体字:\footnotetext{Shell}(也称为壳层)在计算机科学中指"为用户提供用户界面"的软件,
通常指的是命令行界面的解析器。
一般来说,这个词是指操作系统中提供访问内核所提供之服务的程序。
Shell也用于泛指所有为用户提供操作界面的程序,也就是程序和用户交互的层面。
因此与之相对的是内核(英语:Kernel),内核不提供和用户的交互功能。
% 测试不同的URL处理方式:

% 方法1:使用URL编码
from: \url{https://zh.wikipedia.org/wiki/%e6%ae%bc%e5%b1%a4}
from: \url{https://zh.wikipedia.org/wiki/壳层}

% 方法2:使用英文URL
from: \url{https://en.wikipedia.org/wiki/Shell_(computing)}

% 方法3:使用纯文本显示
from: https://zh.wikipedia.org/wiki/殼層

混合繁简体:这是简体,這是繁體,混合显示测试。

\section{行内代码测试}

这个\latexcmd{RequirePackage\{graphicx\}}是行内代码
\bibliography{bib/test.bib}
\end{document}