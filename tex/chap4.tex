\chapter{文献的引用}
\label{chap:chap4}

本项目基于packyan\cite{2020njust_thesis_master}的项目,
采用zepinglee\cite{2025zepinglee}的项目提供GB/T 7714—2015文献引用宏包,
根据《南京理工大学博士、硕士学位论文撰写格式(2014版)》\cite{2014njust_thesis},\\
对\ilcode{gbt7714-numerical.bst}进行调整,
并创建名为\ilcode{gbt7714-numerical_njust.bst}的文件。
所做的修改日志放在了\ilcode{CHANGELOG.MD}中。

在\ilcode{bib/}文件夹中保留了修改前的\ilcode{gbt7714-numerical.bst}
文件,方便进行调试。
未来考虑更好的更新方式以方便与zepinglee\cite{2025zepinglee}仓库 保持同步更新。


\section{文献引用测试}
\label{sec:chinese_citations}

为了测试中文文献的引用效果,本节将展示各种类型的文献引用。
在2014版的标准中,仅对论文类和著作类进行了规定,其余按照
GBT7714-2015\cite{gbt7714-2015}进行调整。表\ref{table:bib_keyword}
来自zepinglee\cite{2025zepinglee}仓库,其中带 “*” 的类型不是 BibTeX 的标准文献类型。

\begin{table}[htbp]
\centering
\caption{文献类型与BibTeX条目类型对照}
\label{table:bib_keyword}
\begin{tabular}{lll}
\hline
文献类型 & 标识代码 & Entry Type \\
\hline
普通图书 & M & \ilcode{@book} \\
图书的析出文献 & M & \ilcode{@incollection} \\
会议录 & C & \ilcode{@proceedings} \\
会议录的析出文献 & C & \ilcode{@inproceedings} 或 \ilcode{@conference} \\
汇编 & G & \ilcode{@collection}$^*$ \\
报纸 & N & \ilcode{@newspaper}$^*$ \\
期刊的析出文献 & J & \ilcode{@article} \\
学位论文 & D & \ilcode{@mastersthesis} 或 \ilcode{@phdthesis} \\
报告 & R & \ilcode{@techreport} \\
标准 & S & \ilcode{@standard}$^*$ \\
专利 & P & \ilcode{@patent}$^*$ \\
数据库 & DB & \ilcode{@database}$^*$ \\
计算机程序 & CP & \ilcode{@software}$^*$ \\
电子公告 & EB & \ilcode{@online}$^*$ \\
档案 & A & \ilcode{@archive}$^*$ \\
舆图 & CM & \ilcode{@map}$^*$ \\
数据集 & DS & \ilcode{@dataset}$^*$ \\
其他 & Z & \ilcode{@misc} \\
\hline
\end{tabular}
\end{table}

\subsection{期刊文章引用}

根据论文类编排内容及顺序一节,应该遵从:

序号.作者姓名.论文题名[J].杂志名.出版年份(期):论文所在页码

例:曹希仁. 离散事件动态系统[J]. 自动化学报. 1985(4):438-446\cite{850419}

Vaswani A, Shazeer N, Parmar N, et al. Attention is all you need[J]. Advances in neural
information processing systems, 2017, 30.\cite{vaswani2017attention}

\subsection{图书引用}

根据著作类编排中内容及顺序一节,应该遵从:

序号.作者姓名.书名[M].版本.出版地:出版者,出版年

例:

严士健. 测试与概率[M]. 第1版. 北京:北京师范大学出版社, 1994\cite{严士健2003测度与概率}

Morton L T. Use of medical literature[M]. 2nd ed. London: Butter-worths, 1977\cite{morton1977use}

\subsection{会议论文引用}


例:

人工智能技术在智能制造领域的应用研究日益深入\cite{wang2021ai}。
相关研究成果为工业4.0的实现提供了技术支撑。

Machine learning algorithms have shown significant improvements in natural language processing tasks\cite{smith2021nlp}。
These advances have opened new possibilities for automated text analysis.

\subsection{学位论文引用}


例:

基于深度学习的图像识别算法研究\cite{liu2020thesis}为计算机视觉领域
提供了新的解决方案。

Advanced neural network architectures for computer vision applications\cite{johnson2020thesis} have demonstrated remarkable performance improvements in object detection tasks.

\subsection{专利引用}


例:

神经网络在图像处理方面的专利技术\cite{patent2021cn}展示了
人工智能技术在知识产权保护方面的重要性。

Novel machine learning algorithms for autonomous vehicle navigation\cite{patent2021us} represent significant advances in transportation technology.

\subsection{技术报告引用}


例:

人工智能技术的发展现状和趋势在相关报告中得到了详细分析\cite{report2021}。

Comprehensive analysis of quantum computing applications in cryptography\cite{report2021quantum} provides insights into future security challenges.

\subsection{标准引用}


例:

人工智能术语的标准化工作\cite{standard2020}为行业发展提供了规范基础。

International standards for data privacy and security\cite{iso2021privacy} establish guidelines for protecting sensitive information in digital systems.

\subsection{报纸文章引用}


例:

人工智能技术在制造业转型升级中的作用\cite{newspaper2021}得到了广泛关注。

Breakthrough in renewable energy storage technology\cite{newspaper2021energy} has attracted significant attention from industry leaders and policymakers.


\subsection{电子资源引用}

例:

在线资源为研究者提供了便捷的信息获取渠道\cite{online2021}。

Open-source software repositories\cite{github2021ml} provide valuable resources for machine learning researchers and practitioners worldwide.

\subsection{论文集引用}

例:

人工智能前沿技术的研究成果\cite{collection2020}汇集了该领域的最新进展。

Recent advances in computational biology\cite{collection2021bio} demonstrate the interdisciplinary nature of modern scientific research.

\subsection{手册引用}

例:

深度学习框架的使用手册\cite{manual2021}为开发者提供了详细的技术指导。

Comprehensive user guide for distributed computing systems\cite{manual2021distributed} offers practical implementation strategies for large-scale applications.
