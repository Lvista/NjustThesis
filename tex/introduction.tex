\chapter{绪论}
\label{chap:introduction}

本模板是南京理工大学硕士论文的Latex模版。
为了方便各位同学使用,并根据自身需求添加或修改配置,在模版正文中加入说明。

\section{相关工作}

关于文献引用的模板,已有多位前辈作出了贡献。2011年不知名前辈\cite{2011njust_thesis}发布了模版,
2015年用户jiec827\cite{jiec827}基于前者和国科大学发布了模版,
2022年用户JunMa11\cite{2020njust_thesis_PhD}记录了博士毕业之路,
同时发布了基于jiec827版的Overleaf版本的模版\cite{2020njust_thesis_PhD_overleaf},
同年用户packyan\cite{2020njust_thesis_master}发布了新的硕士论文模版,修正了中文引用的格式,并删除了一些无用的文件。
2024年TingweiWang\cite{2024njust_thesis_overleaf}对JunMa11版的Overleaf模版进行了更新,
并在CSDN上对此模版进行讨论\cite{2024njust_thesis_csdn}。

\section{快速开始}

对于熟悉Latex写作的用户,可遵照下面说明开始:
\begin{itemize}
    \item \ilcode{myThesis.tex}:这是总的入口文件,修改该文件管理生成的部分
    \item \ilcode{sty/custom.tex}:请将所有包和设定放在这个文件
    \item \ilcode{tex/}:这个文件夹包含所有正文前和正文的tex文件
\end{itemize}

对于一般用户仅通过修改以上部分就可以实现论文的撰写。
如果对Latex环境和包的用法不了解的,可阅读本章剩余章节和第\ref{chap:chap2}章。
对于想要修改样式的用户,可跳转第\ref{chap:chap3}章。

\section{编译环境}
\label{sec:compliment_evn}

在进入格式修改和文章的撰写之前,首先需要一个正常编译的环境。

\LaTeX 有多个发行版,下面的表格总结了几个发行版的特点,各位可按需选择。
考虑到学术论文写作的环境,推荐使用TexLive进行撰写,撰写本文时使用的也是TexLive。

Note: TexLive在MacOS平台上会被安装在各个角落,
完全卸载\footnote{指消除所有相关文件而不留痕迹}极为困难,
因此并不推荐在MacOS平台使用TexLive。

\begin{table}[htbp]
\centering
\caption{操作系统支持}
\label{tab:sys_support}
\begin{tabular}{lcccc}
\toprule
\textbf{发行版} & \textbf{平台} & \textbf{大小} & \textbf{更新方式} & \textbf{适合用户} \\
\midrule
TeX Live & 全平台 & ~7GB & 年更 & 专业用户、研究者 \\
MiKTeX & 主要Windows & ~200MB+ & 滚动更新 & Windows用户、初学者 \\
MacTeX & macOS & ~4.5GB & 年更 & macOS用户 \\
TinyTeX & 全平台 & ~100MB & 按需 & R用户、轻量用户 \\
Overleaf & 在线 & - & 自动 & 协作项目、临时使用 \\
\bottomrule
\end{tabular}
\end{table}

\subsection{TexLive环境}

由于网上关于TexLive的安装教程众多,在这里就不多赘述。

在Shell\footnotemark 环境中进行下面操作,
在Windows上,可用Command Prompt,
在Limux/MacOS中,可用Terminal。

\footnotetext{Shell(也称为壳层)在计算机科学中指“为用户提供用户界面”的软件,
通常指的是命令行界面的解析器。
一般来说,这个词是指操作系统中提供访问内核所提供之服务的程序。
Shell也用于泛指所有为用户提供操作界面的程序,也就是程序和用户交互的层面。
因此与之相对的是内核(英语:Kernel),内核不提供和用户的交互功能。
from: \url{https://zh.wikipedia.org/wiki/\%e6\%ae\%bc\%e5\%b1\%a4}}

\begin{itemize}
    \item 通过以下方式确认安装正确:

\begin{lstlisting}[style=myshell,caption=系统管理脚本]
#!/bin/bash
latex --version
\end{lstlisting}

    \item 本模版库中有一个测试文件\texttt{test.tex},可用于测试
    
\begin{lstlisting}[style=myshell,caption=基本测试]
#!/bin/bash
cd ~/njust_latex
xelatex test.tex
\end{lstlisting}

    \item 如果以上测试正确,则测试模版文件

\begin{lstlisting}[style=myshell,caption=模版测试]
#!/bin/bash
cd ~/njust_latex
xelatex myThesis.tex
\end{lstlisting}

\end{itemize}

\section{文件结构}

本模板项目采用模块化设计,主要文件结构如下:

\begin{itemize}

\item \texttt{myThesis.tex}: 主入口tex文件,包含文档类声明、包引用和内容组织
\item \texttt{README.MD}: 项目自述文件,包含使用说明和版本信息
\item \texttt{test.tex}: 基本测试tex文件,用于验证模板功能
\item \texttt{CHANGELOG.MD}: 项目更新日志文件
\item \texttt{.latexmkrc}: latexmk配置文件,用于自动化编译

\item \texttt{tex/}: 论文内容目录
    \begin{itemize}
        \item \texttt{cover.tex}: 封面页
        \item \texttt{abstract.tex}: 摘要
        \item \texttt{introduction.tex}: 引言
        \item \texttt{chap1.tex}: 第一章(样式设置)
        \item \texttt{chap2.tex}: 第二章(图形和包说明)
        \item \texttt{chap3.tex}: 第三章(文献引用)
        \item \texttt{chap4.tex}: 第四章(用户内容)
        \item \texttt{future.tex}: 未来工作
        \item \texttt{thanks.tex}: 致谢
        \item \texttt{appendix.tex}: 附录
        \item \texttt{publications.tex}: 发表论文
        \item \texttt{statement.tex}: 声明
        \item \texttt{sybolname.tex}: 符号表
    \end{itemize}

\item \texttt{bib/}: 参考文献目录
    \begin{itemize}
        \item \texttt{gbt7714-numerical\_njust.bst}: GBT7714-2015数字引用样式文件(南理工定制版)
        \item \texttt{myThesisRefs.bib}: 参考文献数据库
        \item \texttt{test.bib}: 测试用参考文献
    \end{itemize}

\item \texttt{sty/}: 样式文件目录
    \begin{itemize}
        \item \texttt{njustThesis.cls}: 南京理工大学论文文档类
        \item \texttt{njustThesis.cfg}: 文档类配置文件
        \item \texttt{commons.sty}: 通用设置包
        \item \texttt{custom.sty}: 用户自定义命令包
        \item \texttt{manual.sty}: 手册样式包
    \end{itemize}

\item \texttt{img/}: 图片资源目录
    \begin{itemize}
        \item \texttt{eg\_png.png}: 示例图片
        \item \texttt{logo/}: 学校标志目录
            \begin{itemize}
                \item \texttt{njust\_logo.eps}: 南理工标志
                \item \texttt{njust\_name.eps}: 南理工名称
                \item \texttt{njust.eps}: 南理工完整标志
                \item \texttt{njust-eps-converted-to.pdf}: 转换后的PDF格式标志
            \end{itemize}
    \end{itemize}

\end{itemize}




