\chapter{样式设置}
\label{chap:chap3}

为了方便相关人员对本模版进行调试,同时也为用户提供指引,
对南京理工论文模版的样式构成和配置进行说明。
\begin{itemize}
  \item 带有 \redstar 表示用户需要根据自身需求自定义的部分。
  \item 带有 \prohibit 表示用户不应该修改的部分
\end{itemize}


本文尽量以《南京理工大学博士、硕士学位论文撰写格式(2014版)》\cite{2014njust_thesis},
中的说明顺序对相应的配置进行说明,使用搜索功能对相关代码进行定位。


\section{页面布局}

定义:\ilcode{sty/njustThesis.cls >>> page layout}

使用低层参数对页面布局进行设置,每个参数的含义可参考附录A:\ref{appd_A:layout}。
为了方便,以参数\ilcode{\hoffest}和\ilcode{\voffset}作为正文边界,
但实质上,正文区域(Body)的上沿到纸上沿距离$D_{topspace}$为:

$D_{topspace}$ = one inch + \ilcode{\voffest} + \ilcode{\topmargin} + 
\ilcode{\headheight} + \ilcode{\headsep}

为了方便计算,令:

\ilcode{\topmargin} + \ilcode{\headheight} + \ilcode{\headsep} = 0

即:

\ilcode{\topmargin} = -1 * (\ilcode{\headheight} + \ilcode{\headsep})

所以你能看到\ilcode{\topmargin}是取负。

比如在这里,令页眉上沿离纸上沿距离为20mm,而$D_{topspace}$按照规定为30mm,
则\ilcode{\topmargin}=20-30=-10mm,页眉本体区域高\ilcode{\headheight}=15pt,
页眉底到正文上沿\ilcode{\headsep}=10mm-15pt=4.7mm,注意单位的转换。
($1pt\approx 0.35146mm$)

另外,保留了单双面装订的选择,并使用\ilcode{@twoside}对单双面装订分别进行配置。

\begin{lstlisting}[style=mylatex]
%% myThesis.tex

%% add oneside option
\documentclass[UTF8,AutoFakeBold,oneside]{sty/njustThesis}
\end{lstlisting}

\section{字体和间距}

字体和间距的设置被放置在多个地方,主要分为标题和正文。

\subsection{正文}

定义:\ilcode{njustThesis.cls >>> loadclass infomation}

\ilcode{\zihao}这个选项控制了整个文档的字体基准,同时也是正文字号。
这里使用正文小四号字体,(\ilcode{\zihao=-4})具体参考\CTeXlogo 文档中关于字号的说明


\subsection{标题}
定义:\ilcode{sty/njustThesis.cfg >>> the chapter title format}及后面

\begin{itemize}
    \item \ilcode{\songti}在不同系统对应不同的字库,会有差别,参考\CTeXlogo
    \item 字号的定义具体参考\CTeXlogo,由于 其中没有四号字体,所以这里直接使用\ilcode{\zihao}
    \item \ilcode{\beforeskip}和\ilcode{\aftereskip}用于设置前后间距,
            单位磅就等于pt
\end{itemize}

\section{页眉和页脚}

定义:\ilcode{sty/njustThesis.cls >>> page setting}

\begin{itemize}
    \item 使用小五号宋体,注意,从中文摘要开始应用页眉页脚,这也是为何将这一设置放在中文摘要前的原因。
    \item 通过修改变量\ilcode{\def\NJUST@value@pageDegree}\redstar 修改封面标题和页眉
\end{itemize}

\section{封面}

定义:\ilcode{tex/cover.tex}\redstar

定义:\ilcode{sty/njustThesis.cls >>> make chinese titlepage}及后续


\begin{itemize}
    \item 按顺序从上到下排列元素,其中下划线使用\ilcode{ulem}包,并自定义命令:
\end{itemize}

\begin{lstlisting}[style=mylatex]
%% underline - check if ulem package is loaded
\@ifpackageloaded{ulem}{}{%
  \RequirePackage[normalem]{ulem}%
}
\def\NJUST@underline[#1]#2{%
  \uline{\hbox to #1{\hfill#2\hfill}}}
\def\NJUSTunderline{\@ifnextchar[\NJUST@underline\uline}
\end{lstlisting}

\begin{itemize}
    \item “X士学位论文”大标题使用36号标准字体,而不是学校文件要求的32号
\end{itemize}

\section{声明}
定义:\ilcode{tex/statment} \prohibit

定义:\ilcode{sty/njustThesis.cls >>> make statement titlepage}

声明来自学校相关文件\cite{2014njust_thesis},按要求用户不能修改

\section{摘要和关键字}

定义:\ilcode{tex/abstract} \redstar

定义:\ilcode{sty/njustThesis.cls >>> make abstract & keywords}

根据文件要求如下:
\begin{itemize}
  \item 一般不用图表、化学结构式和非公知公用的符号和术语
  \item 硕士学位论文摘要中文字数400\textasciitilde600个字,
  博士学位论文摘要中文字数800\textasciitilde1000个字
  \item 关键字一般选取3\textasciitilde8个
\end{itemize}

\section{目次页}

定义:\ilcode{sty/njustThesis.cls >>> make contents}

通过\ilcode{\tableofcontents}命令生成目录,通过\ilcode{{\Nchapter}}
命令添加目录
\begin{lstlisting}[style=mylatex]
  {\centering \Nchapter{附\NJUSTspace 录 A}}
\end{lstlisting}
可选设置
\begin{itemize}
  \item \ilcode{\hypersetup{linkcolor=blue}}:设置目录字体色,默认蓝色
  \item \ilcode{ \vskip 0.0em \@plus\p@}:修改章节间的间隔
  \item \ilcode{\renewcommand{\@dotsep}{4.5}}:点的间隔
\end{itemize}

\section{图表目录}

定义:\ilcode{sty/njustThesis.cls >>> figure and table content}

\section{图和表设定}

定义:\ilcode{sty/njustThesis.cls >>> figure and table setting}

\section{正文}

根据要求,
博士学位论文一般为5~10万字,硕士学位论文一般为3~5万

\section{致谢}

根据要求,可在正文后对下列方面致谢:
\begin{itemize}
  \item 对资助或支持学位论文工作的国家科学基金、奖学金基金、合作单位、组织或个人;
  \item 对指导、协助完成学位论文工作的组织或个人;
  \item 对在做学位论文工作中提出建议和意见的人;
  \item 对给予转载和引用权的资料、图片、文献、研究思想和设想的所有者;
  \item 对其他应感激的组织和个人
\end{itemize}

\section{参考文献}

关于这部分的说明放在第\ref{chap:chap3}章,

\section{附录}

定义:\ilcode{tex/appendix.tex}\redstar

定义:\ilcode{tex/publication.tex}\redstar

定义:\ilcode{sty/njustThesis.cls >>> 7. appendix}

附录不是必须的,通常把符号说明等放附录A,学术成果放在附录B