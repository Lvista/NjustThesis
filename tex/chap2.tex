\chapter{包使用参考}
\label{chap:chap2}

\section{LaTeX包简要说明}
\label{sec:package_description}

本节简要介绍\ilcode{sty/custom.sty}模板中使用的主要LaTeX包及其功能。

\subsection{字体和排版包}

\begin{itemize}
    \item \ilcode{fontspec}: XeLaTeX字体选择包,支持系统字体和OpenType特性
    \item \ilcode{siunitx}: 科学单位包,用于正确排版物理量和单位
    \item \ilcode{physics}: 物理符号包,提供常用物理符号和运算符
\end{itemize}

\subsection{数学和科学包}

\begin{itemize}
    \item \ilcode{amsmath, amssymb, amstext}: AMS数学包套件,提供高级数学功能
    \item \ilcode{mathrsfs}: 数学花体字体包
    \item \ilcode{chemformula}: 化学公式包,用于排版化学方程式
    \item \ilcode{algorithm, algpseudocode}: 算法排版包,用于伪代码和算法描述
\end{itemize}

\subsection{图形和表格包}

\begin{itemize}
    \item \ilcode{graphicx}: 图形插入包,支持多种图像格式
    \item \ilcode{subcaption}: 子图包,用于创建子图和图形矩阵
    \item \ilcode{overpic}: 图像叠加包,可在图像上叠加LaTeX元素
    \item \ilcode{multirow}: 表格多行包,用于复杂表格排版
    \item \ilcode{booktabs}: 专业表格包,提供高质量表格线条
\end{itemize}

\subsection{代码和列表包}

\begin{itemize}
    \item \ilcode{listings}: 代码高亮包,支持多种编程语言
    \item \ilcode{xcolor}: 颜色包,提供颜色定义和文本着色
\end{itemize}

\subsection{超链接和引用包}

\begin{itemize}
    \item \ilcode{hyperref}: 超链接包,提供PDF内部链接和书签
    \item \ilcode{gbt7714}: 中文参考文献包,符合GB/T 7714-2015标准
\end{itemize}

\subsection{浮动体和标题包}

\begin{itemize}
    \item \ilcode{placeins}: 浮动体控制包,防止浮动体跨越章节
    \item \ilcode{caption}: 标题包,自定义图表标题格式
\end{itemize}

\subsection{包配置说明}

\section{数学公式}
\label{sec:mathematics}

这里是一些公式使用参考

\subsection{基础数学公式}

行内公式:爱因斯坦质能方程 $E = mc^2$,薛定谔方程 $\hat{H}\psi = E\psi$。

显示公式:
\begin{equation}
    \nabla \cdot \mathbf{E} = \frac{\rho}{\varepsilon_0}
\end{equation}

麦克斯韦方程组:
\begin{align}
    \nabla \cdot \mathbf{E} &= \frac{\rho}{\varepsilon_0} \\
    \nabla \cdot \mathbf{B} &= 0 \\
    \nabla \times \mathbf{E} &= -\frac{\partial \mathbf{B}}{\partial t} \\
    \nabla \times \mathbf{B} &= \mu_0\mathbf{J} + \mu_0\varepsilon_0\frac{\partial \mathbf{E}}{\partial t}
\end{align}

\subsection{复杂数学表达式}

傅里叶变换:
\begin{equation}
    \mathcal{F}\{f(t)\} = F(\omega) = \int_{-\infty}^{\infty} f(t) e^{-i\omega t} dt
\end{equation}

矩阵运算:
\begin{equation}
    \mathbf{A} = \begin{bmatrix}
        a_{11} & a_{12} & \cdots & a_{1n} \\
        a_{21} & a_{22} & \cdots & a_{2n} \\
        \vdots & \vdots & \ddots & \vdots \\
        a_{m1} & a_{m2} & \cdots & a_{mn}
    \end{bmatrix}, \quad
    \det(\mathbf{A}) = \sum_{\sigma \in S_n} \operatorname{sgn}(\sigma) \prod_{i=1}^n a_{i,\sigma(i)}
\end{equation}

\clearpage
\section{算法描述}
\label{sec:algorithms}

\begin{algorithm}
\caption{梯度下降算法}
\label{alg:gradient_descent}
\begin{algorithmic}[1]
\Require 学习率 $\eta$,最大迭代次数 $T$,目标函数 $f(\mathbf{x})$
\Ensure 最优解 $\mathbf{x}^*$
\State 初始化 $\mathbf{x}_0$ 和 $t \leftarrow 0$
\While{$t < T$ \textbf{and} $\|\nabla f(\mathbf{x}_t)\| > \epsilon$}
    \State 计算梯度 $\mathbf{g}_t \leftarrow \nabla f(\mathbf{x}_t)$
    \State 更新参数 $\mathbf{x}_{t+1} \leftarrow \mathbf{x}_t - \eta \mathbf{g}_t$
    \State $t \leftarrow t + 1$
\EndWhile
\State \Return $\mathbf{x}_t$
\end{algorithmic}
\end{algorithm}

\begin{algorithm}
\caption{K-均值聚类算法}
\label{alg:kmeans}
\begin{algorithmic}[1]
\Require 数据集 $X = \{\mathbf{x}_1, \ldots, \mathbf{x}_n\}$,聚类数 $k$
\Ensure 聚类中心 $C = \{\mathbf{c}_1, \ldots, \mathbf{c}_k\}$,簇分配
\State 随机初始化聚类中心 $C^{(0)}$
\State $t \leftarrow 0$
\Repeat
    \For{$i = 1$ \textbf{to} $n$}
        \State $z_i \leftarrow \arg\min_j \|\mathbf{x}_i - \mathbf{c}_j^{(t)}\|^2$
    \EndFor
    \For{$j = 1$ \textbf{to} $k$}
        \State $\mathbf{c}_j^{(t+1)} \leftarrow \frac{1}{|S_j|} \sum_{\mathbf{x} \in S_j} \mathbf{x}$
    \EndFor
    \State $t \leftarrow t + 1$
\Until{收敛}
\State \Return $C^{(t)}$, $\{z_i\}$
\end{algorithmic}
\end{algorithm}

\clearpage
\section{表格设计}
\label{sec:tables}

\subsection{基础表格}

\begin{table}[htbp]
\centering
\caption{实验数据汇总}
\label{tab:experiment_data}
\begin{tabular}{lccc}
\toprule
\textbf{样本} & \textbf{温度 (\si{\celsius})} & \textbf{压力 (\si{\pascal})} & \textbf{产率 (\%)} \\
\midrule
A & 25.0 & 101325 & 85.2 \\
B & 30.5 & 102100 & 88.7 \\
C & 35.2 & 100950 & 82.1 \\
D & 40.1 & 101800 & 91.5 \\
\bottomrule
\end{tabular}
\end{table}

\subsection{复杂表格}

\begin{table}[htbp]
\centering
\caption{材料性能比较}
\label{tab:materials}
\begin{tabular}{lcccc}
\toprule
\multirow{2}{*}{\textbf{材料}} & \multicolumn{2}{c}{\textbf{力学性能}} & \multicolumn{2}{c}{\textbf{热学性能}} \\
\cmidrule(lr){2-3} \cmidrule(lr){4-5}
 & 强度 (\si{\mega\pascal}) & 模量 (\si{\giga\pascal}) & 导热率 (\si{\watt\per\meter\per\kelvin}) & 膨胀系数 (\si{\per\kelvin}) \\
\midrule
钢 & 250 & 200 & 50 & 1.2e-5 \\
铝 & 90 & 70 & 237 & 2.3e-5 \\
钛 & 240 & 110 & 22 & 8.6e-6 \\
\bottomrule
\end{tabular}
\end{table}

\section{化学方程式}
\label{sec:chemistry}

化学反应:
\begin{equation}
    \ch{2 H2 + O2 -> 2 H2O} \quad \Delta H = \SI{-285.8}{\kilo\joule\per\mole}
\end{equation}

酸碱中和反应:
\begin{equation}
    \ch{HCl + NaOH -> NaCl + H2O}
\end{equation}

氧化还原反应:
\begin{equation}
    \ch{2 Al + Fe2O3 -> Al2O3 + 2 Fe}
\end{equation}

\section{代码列表}
\label{sec:code}

\subsection{Python 科学计算代码}

\begin{lstlisting}[style=mypython,caption=数值积分计算, label=code:integration]
import numpy as np
from scipy import integrate

def monte_carlo_integration(f, a, b, num_samples=10000):
    """蒙特卡洛积分方法"""
    x_random = np.random.uniform(a, b, num_samples)
    f_values = f(x_random)
    integral = (b - a) * np.mean(f_values)
    return integral

# 示例:计算正弦函数在[0, π]上的积分
result = monte_carlo_integration(np.sin, 0, np.pi)
print(f"积分结果: {result}")
\end{lstlisting}

\subsection{数据可视化代码}

\begin{lstlisting}[style=mypython, caption=科学数据可视化, label=code:visualization]
import matplotlib.pyplot as plt
import numpy as np

# 生成数据
x = np.linspace(0, 10, 100)
y1 = np.sin(x)
y2 = np.cos(x)

# 创建图表
fig, (ax1, ax2) = plt.subplots(1, 2, figsize=(12, 4))

# 第一个子图
ax1.plot(x, y1, 'b-', label='sin(x)')
ax1.plot(x, y2, 'r--', label='cos(x)')
ax1.set_xlabel('x')
ax1.set_ylabel('y')
ax1.legend()
ax1.grid(True)

# 第二个子图:散点图
x_scatter = np.random.normal(0, 1, 100)
y_scatter = np.random.normal(0, 1, 100)
ax2.scatter(x_scatter, y_scatter, alpha=0.6)
ax2.set_xlabel('X')
ax2.set_ylabel('Y')

plt.tight_layout()
plt.savefig('scientific_plot.png', dpi=300)
plt.show()
\end{lstlisting}

\section{物理公式和符号}
\label{sec:physics}

本节展示符号表中定义符号的使用示例。根据符号表定义,我们使用以下符号:

量子力学算符:
\begin{equation}
    \hat{p} = -i\hbar \frac{\partial}{\partial x}, \quad
    [\hat{x}, \hat{p}] = i\hbar
\end{equation}

相对论能量动量关系,其中$E$表示能量,$c$表示光速:
\begin{equation}
    E^2 = (pc)^2 + (m_0 c^2)^2
\end{equation}

牛顿第二定律,其中$F$表示力,$m$表示质量,$v$表示速度:
\begin{equation}
    F = m \frac{dv}{dt}
\end{equation}

角度相关的三角函数关系,其中$\alpha$和$\beta$为角度符号:
\begin{equation}
    \sin(\alpha + \beta) = \sin\alpha\cos\beta + \cos\alpha\sin\beta
\end{equation}

伽马射线衰减公式,其中$\gamma$表示伽马射线符号:
\begin{equation}
    I = I_0 e^{-\mu x}
\end{equation}
其中$I_0$为初始强度,$\mu$为衰减系数,$x$为穿透距离。

热力学基本关系:
\begin{equation}
    \dd U = T \dd S - P \dd V + \sum_i \mu_i \dd N_i
\end{equation}

\section{图片}
\begin{figure}[htbp]
    \centering
    \includegraphics[width=0.8\textwidth]{img/eg_png.png}
    \caption{示例图片}
    \label{fig:example}
\end{figure}
位置:\ilcode{sty/custom.sty >>> Graphic (figure)}
\begin{itemize}
    \item \ilcode{\RequirePackage{graphicx}}: 图像基本类,调用方法如下
\begin{lstlisting}[style=mylatex]
%% 单图
\begin{figure}[htbp]
    \centering
    \includegraphics[width=0.8\textwidth]{img/eg_png.png}
    \caption{示例图片}
    \label{fig:example}
\end{figure}
\end{lstlisting}
    \item \ilcode{\RequirePackage{subcaption}}: 现代的子图包,用于创建子图
\begin{lstlisting}[style=mylatex]
%% 子图
\begin{figure}[htbp]
    \centering
    \begin{subfigure}[b]{0.45\textwidth}
        \includegraphics[width=\textwidth]{img/sub1.png}
        \caption{子图1}
        \label{fig:sub1}
    \end{subfigure}
    \hfill
    \begin{subfigure}[b]{0.45\textwidth}
        \includegraphics[width=\textwidth]{img/sub2.png}
        \caption{子图2}
        \label{fig:sub2}
    \end{subfigure}
    \caption{包含两个子图的示例}
    \label{fig:subfigures}
\end{figure}
%% 图形矩阵
\begin{figure}[htbp]
    \centering
    \begin{subfigure}[b]{0.3\textwidth}
        \includegraphics[width=\textwidth]{img/matrix1.png}
        \caption{图1}
    \end{subfigure}
    \hfill
    \begin{subfigure}[b]{0.3\textwidth}
        \includegraphics[width=\textwidth]{img/matrix2.png}
        \caption{图2}
    \end{subfigure}
    \hfill
    \begin{subfigure}[b]{0.3\textwidth}
        \includegraphics[width=\textwidth]{img/matrix3.png}
        \caption{图3}
    \end{subfigure}
    \caption{3×1图形矩阵示例}
    \label{fig:subfigmat}
\end{figure}
\end{lstlisting}
    \item \ilcode{\RequirePackage{overpic}}: 在图像上叠加LaTeX元素
\begin{lstlisting}[style=myshell]
%% 图上叠加文本
\begin{figure}[htbp]
    \centering
    \begin{overpic}[width=0.8\textwidth]{img/background.png}
        \put(50,50){\textcolor{red}{\Large 叠加文本}}
        \put(100,100){\textcolor{blue}{\Large 坐标(100,100)}}
    \end{overpic}
    \caption{在图像上叠加LaTeX元素}
    \label{fig:overpic}
\end{figure}
\end{lstlisting}
    \item \ilcode{\DeclareGraphicsExtensions{.pdf,.png,.jpg}}: 声明支持的图像格式
\end{itemize}






